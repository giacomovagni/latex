\documentclass[doc, twocolumn]{apa}[12pt]
\usepackage{amsmath}
\usepackage{url}

\usepackage{graphicx,epstopdf}
\usepackage{pdfpages}
\usepackage{epigraph}
\usepackage{colortbl}
\usepackage{lscape}
\usepackage{alltt}
\usepackage{xcolor,colortbl}
\usepackage{apacite}
\usepackage{amsfonts}
\usepackage{amssymb}
\usepackage{epigraph}
\usepackage{multirow}
\usepackage{graphicx}
\usepackage{xcolor,colortbl}
\usepackage{color}
\usepackage{subfigure}
\usepackage[maxfloats=40]{morefloats}
%\usepackage{titlesec}
\usepackage{longtable}
\usepackage{lscape}
\usepackage{lipsum}
%\usepackage{titlepic}

\pagestyle{plain}

\usepackage{ebgaramond}

\usepackage[width=17.00cm, height=20.00cm]{geometry}

\onecolumn

\journal{Working Paper. May 2016.}
\author{Giacomo Vagni}

\setlength{\columnsep}{1.5cm}

\title{All You Need is Love \\ \emph{Searching for Causal Effects of Spouse on Enjoyment}}
\affiliation{PhD Student \\ University of Oxford \\ Nuffield College}

\vspace{1.2cm}

\abstract{This paper explores the causal effects of the presence of the spouse, children and others on individual enjoyment. Using the latest United Kingdom Time Use Survey (2015), this paper demonstrates that the spouse significantly increases the enjoyment of the moment, regardless of the activity performed. The effect holds for both men and women. However, I show that couples with low conjugal satisfaction tend to enjoy the presence of the partner less. Moreover, I have found that for women the average causal effect of children impacts negatively on enjoyment. I also demonstrate that ``others'' (friends, acquaintances, kin, \dots) have a larger (positive) impact on enjoyment than the partner. The paper discusses potential mechanisms and concludes on the relevance of enjoyment for the study of marital quality.} 


\acknowledgements{Giacomo Vagni can be contacted at giacomo.vagni[at]sociology.ox.ac.uk. The author gratefully thanks Professor Gershuny \& Professor Sullivan for their supervision as well as the CTUR team for their support.}

\shorttitle{Transfer of Status - Vagni}
\rightheader{Transfer of Status - Vagni}
\leftheader{Giacomo Vagni}

\begin{document}
	\maketitle
	\newpage 

\twocolumn 

%\input{/Users/giacomovagni/Rprojects/division_of_joy/part1.txt}

The time that a couple can spend together relies on the possibility for partners to synchronise their schedule. As post-industrial societies are moving towards a 24-hour economy~\cite{presser2005working}, work schedules have become harder to coordinate for couples and families~\cite{Lesnard2009-kl}. Some research argues that the problem of coordination has impacted negatively marital quality~\cite{Mills2010-ux}. The mechanism hypothesized is the lack of time available for each other. It is postulate that partners want to spend time together and not being able to do so, can create tensions and frustration and ultimately can lead to lower conjugal satisfaction~\cite{hill1988marital}. However, evidence is mixed concerning the enjoyment of time together. Some research has found that time together is highly enjoyed and that couples actively try to coordinate their schedules~\cite{Sullivan1996-pa, Flood2016-qz}. Others argued that family life has become so stressful that individuals would rather spend time at work rather than time with family members~\cite{Hochschild2012-yd}. Moreover, several studies stressed that enjoyment of time together varied by gender~\cite{Milkie1999-su} and by activity preferences~\cite{Crawford2002-hy}. The aim of this study is to contribute to the debate in applying econometric models on national representative time use data of the UK population. Most research studying time together and marital quality are mainly descriptive in nature. The contribution of this paper is to go a step further and estimate the causal effects of the spouse on individual enjoyment. 
seems very reliable to me, however additional tests are always important. I will use the imputed enjoyment in this study. cts is the following: 


The results strongly support the hypothesis that on average the presence of the partner increases the enjoyment of the moment. However, the interaction between the presence of the spouse and conjugal satisfaction showed some nuances. Interestingly, children have a negative impact on women's enjoyment. There was no effect of children for men. Moreover, results showed that others (friends, acquaintances, kin, \dots) had a strong positive impact on enjoyment, suggesting that time with personal networks is important for the welfare of individuals. Regarding the hypotheses, we can conclude that only hypothesis 2 does not hold. The average causal effect of the partner for women is not lower than for men. The results from the interaction suggest that hypothesis 4 is more subtle than simply stating that happy couples enjoy more their time together. It seems that general satisfaction and enjoyment have a complex relationship and might feed each other. Many factors can cause conjugal dissatisfaction. One of these could be the lack of quality time with the partner. Therefore, it can make sense that really dissatisfied couples still enjoy greatly their time together. However, these two phenomenon are distinct. While conjugal satisfaction is a more general and abstract process, enjoyment is located in a more precise and particular temporality. It is reasonable to assume that conjugal satisfaction would follow a function such as 


\begin{equation}	\gamma_{i} = \sum \lambda_{it} | D = \end{equation}

\bibliography{/Users/giacomovagni/Rprojects/division_of_joy/bib/bib.bib}
\bibliographystyle{apacite}

\clearpage
\onecolumn


\end{document}


